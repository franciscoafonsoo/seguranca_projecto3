%formatação do documento e tipo do documento
\documentclass[12pt, a4paper, twoside]{report} %inicio do doc

%pacotes de extensões
\usepackage[portuges]{babel} %pkg da lingua portugues
\usepackage[latin1,utf8]{inputenc} %pkg da lingua portugues
\usepackage{verbatim} %pkg para escrever sem formataçao
\usepackage{color} %usar cores nas letras
\usepackage{graphicx} %usar imagens no doc
\usepackage[table,xcdraw]{xcolor}
\usepackage{makeidx}
\usepackage{anysize} % para formatar o tamanho do documento
\usepackage{footnote}
\usepackage{listings}

\makeindex

\begin{document}

\title{%
	\textbf{Segurança Informática}\\ 
	\large Relatório Terceiro Projecto, Grupo 001
}

\author{%
Francisco Pires, nº 44314 \\
Pedro Neves, nº 45787 \\
Tiago Maurício, nº 45105 \\
}

\date{\today}
\maketitle

\chapter{Parte I: iptables}

\section{Regras aplicadas}

As seguintes regras foram aplicadas por esta ordem antes de serem feitos os testes:

\begin{enumerate}

\item ligações já estabelecida devem ser aceites

\texttt{\$ sudo iptables -A INPUT -m state --state ESTABLISHED,RELATED \
 -j ACCEPT}

\texttt{\$ sudo iptables -A OUTPUT -m state --state ESTABLISHED,RELATED \
 -j ACCEPT}

\item A máquina responde a pings apenas com origem nas máquinas da sua sub-rede local (com máscara 255.255.254.0)

\texttt{\$ sudo iptables -A INPUT -s 10.101.148.182/23 -p icmp -j ACCEPT}

\item Aceita ligações de clientes com qualquer origem para o servidor \textit{myWhats}

\texttt{\$ sudo iptables -A INPUT -p tcp --dport 23457 -j ACCEPT}

\item Aceita ligações ssh apenas da máquina gcc

\texttt{\$ sudo iptables -A INPUT -s gcc.alunos.di.fc.ul.pt -p tcp \
 --dport ssh -j ACCEPT}

\item A máquina apenas pode fazer ping à máquina gcc.

\texttt{\$ sudo iptables -A OUTPUT -d gcc.alunos.di.fc.ul.pt -p icmp -j ACCEPT}

\clearpage

\item Não impedir o acesso às maquinas descritas no enunciado, para o bom funcionamento das máquinas do laboratório.


\texttt{\$ sudo iptables -A INPUT -d "10.101.253.11, \
	10.101.253.12, 10.101.253.13, \
	10.121.53.14, 10.121.53.15, \
	10.101.53.16, 10.101.249.63, \
	10.101.85.6, 10.101.85.138, \
	10.101.85.18, 10.101.148.1, \
	10.101.85.134" \ -j ACCEPT}

\texttt{\$ sudo iptables -A OUTPUT -s "10.101.253.11, \
	10.101.253.12, 10.101.253.13, \
	10.121.53.14, 10.121.53.15, \
	10.101.53.16, 10.101.249.63, \
	10.101.85.6, 10.101.85.138, \
	10.101.85.18, 10.101.148.1, \
	10.101.85.134" \ -j ACCEPT}

\item Não filtrar o tráfego do dispositivo de loopback

\texttt{\$ sudo iptables -A INPUT -i lo -j ACCEPT}

\texttt{\$ sudo iptables -A OUTPUT -o lo -j ACCEPT}

\item Não permitir mais nenhuma ligação na table \textit{input} e \textit{output} para alem das acima definidas

\texttt{\$ sudo iptables -A INPUT  -j DROP}

\texttt{\$ sudo iptables -A OUTPUT -j DROP}

\end{enumerate}

\section{Resultado das regras aplicadas (iptables -L)}

\begin{lstlisting}
Chain INPUT (policy ACCEPT)
target     prot opt source               destination
ACCEPT     all  --  anywhere             anywhere       \
      state RELATED,ESTABLISHED
ACCEPT     icmp --  10.101.148.0/23      anywhere
ACCEPT     tcp  --  anywhere             anywhere        \
     tcp dpt:23457
ACCEPT     tcp  --  gcc.alunos.di.fc.ul.pt  anywhere      \
       tcp dpt:ssh
ACCEPT     all  --  anywhere             10.101.253.11
ACCEPT     all  --  anywhere             10.101.253.12
ACCEPT     all  --  anywhere             10.101.253.13
ACCEPT     all  --  anywhere             10.121.53.14
ACCEPT     all  --  anywhere             10.121.53.15
ACCEPT     all  --  anywhere             10.101.53.16
ACCEPT     all  --  anywhere             10.101.249.63
ACCEPT     all  --  anywhere             iate.alunos.di.fc.ul.pt
ACCEPT     all  --  anywhere             falua.di.fc.ul.pt
ACCEPT     all  --  anywhere             nemo.alunos.di.fc.ul.pt
ACCEPT     all  --  anywhere             submarino.alunos.di.fc.ul.pt
ACCEPT     all  --  anywhere             farol.alunos.di.fc.ul.pt
ACCEPT     all  --  anywhere             anywhere
DROP       all  --  anywhere             anywhere
\end{lstlisting}
\clearpage
\begin{lstlisting}
Chain FORWARD (policy ACCEPT)
target     prot opt source               destination

Chain OUTPUT (policy ACCEPT)
target     prot opt source               destination
ACCEPT     all  --  anywhere             anywhere        \
     state RELATED,ESTABLISHED
ACCEPT     icmp --  anywhere             gcc.alunos.di.fc.ul.pt
ACCEPT     all  --  anywhere             10.101.253.11
ACCEPT     all  --  anywhere             10.101.253.12
ACCEPT     all  --  anywhere             10.101.253.13
ACCEPT     all  --  anywhere             10.121.53.14
ACCEPT     all  --  anywhere             10.121.53.15
ACCEPT     all  --  anywhere             10.101.53.16
ACCEPT     all  --  anywhere             10.101.249.63
ACCEPT     all  --  anywhere             iate.di.fc.ul.pt
ACCEPT     all  --  anywhere             falua.di.fc.ul.pt
ACCEPT     all  --  anywhere             nemo.alunos.di.fc.ul.pt
ACCEPT     all  --  anywhere             submarino.alunos.di.fc.ul.pt
ACCEPT     all  --  anywhere             farol.alunos.di.fc.ul.pt
ACCEPT     all  --  anywhere             anywhere
DROP       all  --  anywhere             anywhere
\end{lstlisting}

\clearpage

\section{Testes}

\subsection{Teste da Regra nº 2}

\noindent Realizamos um ping de outra maquina do laboratório para testar esta regra: \\

\noindent Comando: \texttt{\$ ping -c 3 10.101.148.4}

\noindent Resultado:

\begin{lstlisting}
PING 10.101.148.4 (10.101.148.4) 56(84) bytes of data.
64 bytes from 10.101.148.4 (10.101.85.18): icmp_seq=1 ttl=64 time=0.242
64 bytes from 10.101.148.4 (10.101.85.18): icmp_seq=2 ttl=64 time=0.269
64 bytes from 10.101.148.4 (10.101.85.18): icmp_seq=3 ttl=64 time=0.265

--- 10.101.148.4 ping statistics ---
3 packets transmitted, 3 received, 0% packet loss, time 1998ms
rtt min/avg/max/mdev = 0.242/0.258/0.269/0.022 ms
\end{lstlisting}

\subsection{Teste da Regra nº 3}

\noindent Por uma questão de simplicidade, utilizados o modulo \texttt{SimpleHTTPServer} do python para criar um servidor HTTP na mesma porta onde era suposto correr o servidor \textit{myWhats}, e o telnet para fazer um pedido a simular o cliente. Referimos que as respostas por parte dos módulos só servem para verificar que a porta esta aberta e é possível aceder. \\

\noindent Comando no servidor: \texttt{ python -m SimpleHTTPServer 23457}

\noindent Resultado: \\

\noindent \texttt{Serving HTTP on 0.0.0.0 port 23457 ...} \\
\noindent \texttt{10.101.148.182 - - [14/May/2016 17:44:47] code 400, message Bad request syntax ('GET')} \\
\texttt{10.101.148.182 - - [14/May/2016 17:44:47] 'GET' 400 -} \\


\noindent Comando no cliente: \texttt{ telnet 10.101.148.183 23457}

\noindent Resultado: \\

\noindent \texttt{Trying 10.101.148.183...} \\
\noindent \texttt{Connected to 10.101.148.183.} \\
\noindent \texttt{...}

\clearpage

\subsection{Teste da Regra nº 4}

\noindent Não é possível testar esta regra, visto que os alunos não tem autorização para usar o ssh na maquina gcc. Por isso, 
usamos os nossos computados pessoais numa rede local para testar a regra. \\

\noindent ip maquina local: 192.168.1.66

\noindent ip maquina remota (gcc): 192.168.1.82 \\

\noindent Alteração da regra para: \\
\texttt{\$ sudo iptables -A INPUT -s 192.168.1.82 -p tcp \
 --dport ssh -j ACCEPT} \\
\texttt{\$ sudo iptables -A INPUT -j DROP} \\


\noindent Comando para teste: \texttt{ssh 192.168.1.66}

\begin{lstlisting}
sherby@sherby-desktop:~$ ssh 192.168.1.66
sherby@192.168.1.66's password: 
Welcome to Ubuntu 16.04 LTS (GNU/Linux 4.4.0-22-generic x86_64)

 * Documentation:  https://help.ubuntu.com/

0 packages can be updated.
0 updates are security updates.

Last login: Sat May 14 22:18:26 2016 from 192.168.1.82
->  ~ 
\end{lstlisting}

\subsection{Teste da Regra nº 5}

\noindent Realizamos um ping da maquina gcc para testar esta regra: \\

\noindent Comando: \texttt{\$ ping -c 3 10.101.148.4}

\noindent Resultado:

\begin{lstlisting}
PING gcc.alunos.di.fc.ul.pt (10.101.151.5) 56(84) bytes of data.
64 bytes from gcc.alunos.di.fc.ul.pt (10.101.151.5): icmp_seq=1 \
	ttl=64 time=0.341 ms
64 bytes from gcc.alunos.di.fc.ul.pt (10.101.151.5): icmp_seq=2 \
	ttl=64 time=0.403 ms
64 bytes from gcc.alunos.di.fc.ul.pt (10.101.151.5): icmp_seq=3 \
	ttl=64 time=0.333 ms

--- gcc.alunos.di.fc.ul.pt ping statistics ---
3 packets transmitted, 3 received, 0% packet loss, time 2000ms
rtt min/avg/max/mdev = 0.333/0.359/0.403/0.031 ms
\end{lstlisting}

\end{document}
